%%%%%%%%%%%%%%%%%%%%%%%%%%%%%%%%%%%%%%%%%%%%%%%%%%%%%%%%%%%%%%%%%%%%%%%%%%%%%%%%
% Medium Length Graduate Curriculum Vitae
% LaTeX Template
% Version 1.2 (3/28/15)
%
% This template has been downloaded from:
% http://www.LaTeXTemplates.com
%
% Original author:
% Rensselaer Polytechnic Institute 
% (http://www.rpi.edu/dept/arc/training/latex/resumes/)
%
% Modified by:
% Daniel L Marks <xleafr@gmail.com> 3/28/2015
%
% Important note:
% This template requires the res.cls file to be in the same directory as the
% .tex file. The res.cls file provides the resume style used for structuring the
% document.
%
%%%%%%%%%%%%%%%%%%%%%%%%%%%%%%%%%%%%%%%%%%%%%%%%%%%%%%%%%%%%%%%%%%%%%%%%%%%%%%%%

%-------------------------------------------------------------------------------
%	PACKAGES AND OTHER DOCUMENT CONFIGURATIONS
%-------------------------------------------------------------------------------

%%%%%%%%%%%%%%%%%%%%%%%%%%%%%%%%%%%%%%%%%%%%%%%%%%%%%%%%%%%%%%%%%%%%%%%%%%%%%%%%
% You can have multiple style options the legal options ones are:
%
%   centered:	the name and address are centered at the top of the page 
%				(default)
%
%   line:		the name is the left with a horizontal line then the address to
%				the right
%
%   overlapped:	the section titles overlap the body text (default)
%
%   margin:		the section titles are to the left of the body text
%		
%   11pt:		use 11 point fonts instead of 10 point fonts
%
%   12pt:		use 12 point fonts instead of 10 point fonts
%
%%%%%%%%%%%%%%%%%%%%%%%%%%%%%%%%%%%%%%%%%%%%%%%%%%%%%%%%%%%%%%%%%%%%%%%%%%%%%%%%

\documentclass[line,margin,9pt]{res} 
\newsectionwidth{1in}
\topmargin= -0.8in
%\bottommargin= 0in
\oddsidemargin -.5in
\evensidemargin -.75in
\textheight 10.3in
\textwidth=6.5in
%\resumewidth=5in
\itemsep=0in
\parsep=0in


%\usepackage[legalpaper, margin=0.5in]{geometry}
% margin = 1in

% Default font is the helvetica postscript font
\usepackage{helvet}
\usepackage[colorlinks = true,
            linkcolor = blue,
            urlcolor  = blue,
            citecolor = blue,
            anchorcolor = blue]{hyperref}
% Increase text height
\usepackage{hyperref}
\usepackage{fontawesome}
\usepackage[utf8]{inputenc}
\usepackage{dirtytalk}

%\textheight=700pt

\begin{document}
%\newsectionwidth{0.71in}

%-------------------------------------------------------------------------------
%	NAME AND ADDRESS SECTION
%-------------------------------------------------------------------------------
\name{Tyler Andrew Cultice \noindent\hbox to 0.5\textwidth{} (937)631-9267}


\address{Email: tcultice@vols.utk.edu\\\hfil
\href{https://www.linkedin.com/in/tyler-cultice-43265520b/}{\textbf{[LinkedIn]}} \\
\href{https://scholar.google.com/citations?hl=en\&user=9eUFpBwAAAAJ}{\textbf{[Google Scholar]}} %\quad \href{https://sites.google.com/view/md-saif-hassan/}{\textbf{[Personal Site]}
}\hfil

% \href{https://www.researchgate.net/scientific-contributions/Tyler-Cultice-2178791037}{\textbf{[Researchgate]}}

\begin{resume}
\vspace{-10mm}

\section{EDUCATION}
\begin{itemize}
\item \bf{University of Tennessee, Knoxville}, TN-37916, USA\\
\bf GPA 4.00/4.00 \hfill \sl PhD in Computer Engineering\\
\textbf {2021--July 2025}

\item \bf{University of Kentucky}, Lexington, KY-40506, USA\\
\bf GPA 3.879/4.00 \hfill \sl Bachelor of Science in Computer Engineering\\
\textbf {2017--2021}

% \item \bf{Shahid Syed Nazrul Islam College}, Mymensingh-2201, BD\\
% \sl Higher Secondary-School Certificate in Science\\
% \sl GPA \bf 5.00/5.00\\
% \textbf {2013-2015}\\
% \sl Graduation: \bf August,2015.


\end{itemize}

%\section{RESEARCH\\INTERESTS}
%\begin{itemize}
%\item \bf Cybersecurity, 
%\bf Cyberphysical Systems, 
%\bf Firmware Design, 
%\bf Embedded Systems, 
%\bf FPGAs,
%\bf Post-Quantum Cybersecurity, 
%\bf Additive Manufacturing, 
%\bf Quantum Anomaly Detection, 
%\bf Quantum Machine Learning
%\end{itemize}


\section{ENGINEERING\\SKILLS\\\textbf{[Level: Advanced]}}
\begin{itemize}
\item \textbf{Programming Languages:} Python, \textit{Embedded C, C, C++, C\#}, Rust, Verilog/SystemVerilog, VHDL, Java, Javascript, Matlab,\textit{ ARM/x86 Assembly}. \textbf{Design Tools:} Virtuoso, Vivado, Innovus,  \textit{Keil (AMD)}, Quartus. \textbf{Hardware:} \textit{RTOS}, FPGA Design and Synthesis, CUDA, RISC-V. \textbf{Machine Learning \& Quantum Libraries:} Keras, Pytorch, Tensorflow, IBM Qiskit, OpenQASM, Pennylane. \textbf{Security Tools:} NMAP, Metasploit, SQLMap, Netcat. \textbf{Operating System:} Windows, Linux.
\end{itemize}
%-------------------------------------------------------------------------------
% Modify the format of each position
\begin{format}
\title{l}\\
\dates{l}\location{r}\\
\body\\
\end{format}
%-------------------------------------------------------------------------------
% \vspace{.4cm}
% \section{RESEARCH \\ BACKGROUND}
% \begin{itemize}


% \item \textbf{BSc thesis: LULC classification by semantic segmentation of satellite images using FastFCN .}\\
% Supervisor: Dr. A. K. M. Nazrul Islam, Professor, MIST

% \item \textbf{Smart Wheelchair with Medical Sensors and Voice Controlled Navigation: } Worked on this project in a group of 5.It has voice controlled module, temperature sensor, Ultrasonic sensor and GSM module.

% \item \textbf{Doc Scanner for windows PC: }Creates PDF out of pictures. Faster than mobile device. Used computer vision module and is written in python.

% \item \textbf{Satellite Image Segmentation: }Used the Pytorch module to reproduce FastFCN and introduced a grid block to efficiently segment high resolution satellite images.

% \item \textbf{Road Sign Recognizer using open-cv: }Trained a dataset of road signs in a VGG16 architecture.Used it to test various road signs which can help to improve the self driven vehicles.

% \item \textbf{License plate recognition: }Used the YOLO v4 for license plate localization and Tessaract V5 for character recognition

% \item \textbf{Automatic DG placement with the help of Machine learning: }Used KNN for forecasting and predicting bus voltage and line loss to decide on new DG placement.
% \clearpage

% \item \textbf{Automatic OMR detection Using Image Processing: }Detects marked answers and evaluate it in OMR using computer vision. Updates the grades according to the detection algorithm.

% \item \textbf{Bangla Character Recognition with custom CNN: }It could recognise handwritten and printed bangla characters with more than 93\% accuracy. Trained with open-source bangla character datasets on custom made CNN.

% \item \textbf{Eight bit PC with Assembly Language: }It could perform 10 basic commands and was Designed in Proteus and operated by Assembly language.

% \item \textbf{Modelling Non-Verbal Communications for physically challenged:}(Ongoing)


% \end{itemize}

\section{EXPERIENCE}
\begin{itemize}
\item Undergraduate Research Assistant, Dept of ECE, University of Kentucky, Lexington (2019-2021)
\item Teaching Assistant, Dept of EECS, University of Tennessee, Knoxville (2021-Present)
\item \textbf{NSF Graduate Research Fellow}, Dept of EECS, University of Tennessee, Knoxville (2022-Present)

\end{itemize}

\section{DISTINCTIONS AND AWARDS}
\begin{itemize}
\item \textbf{2020 - Schneider Electric Fellowship, University of Kentucky:} \\
Smart-Home Research on ML-based CPS Anomaly Detection in collaboration with SPARK Lab.
\item \textbf{2021 - Upsilon Pi Epsilon (UPE) Award:} \\
IEEE Computer Society Award for Academic Success in Computer Engineering.
\item \textbf{2022--Present - NSF Graduate Research Fellowship Program (GRFP):} \\
Prestigious fellowship awarded to outstanding students with significant contributions to STEM.
\item \textbf{2025 - UTK Gonzalez Outstanding Graduate Teaching Assistant Award: } \\
Award for outstanding commitment, dedication, and notable ambition in EECS academic pursuits.
\end{itemize}

\section{RELEVANT PROJECTS}
\begin{itemize}

\item \textbf{Crystals-KYBER-based Post-Quantum Cryptographic 3D Printing Security -} 
\begin{itemize}
\item Developed a Crystals-KYBER-based post-quantum cryptographic framework in Embedded C for 3D Printer in collaboration with Southeastern Advanced Machine Tools Network (SEAMTN).
\item Engineered a novel \& highly efficient CAN communication tree/graph structure for 3D printers.
\item Built a proof-of-concept 3D printer farm, presented at security conference HOST, with simple API callback structure for plug-and-play.
\item Taught prospective engineers offensive/defensive security for manufacturing networks with Kali.
\end{itemize}

\item \textbf{Post-Quantum Cryptographic Vehicular Security Framework -} 
\begin{itemize}
\item Developed an embedded CAN framework safe from quantum threats for commercial vehicles while adhering to original ISO 11898 protocol specifications.
\item Built embedded testbench for collecting performance benchmarks of this design.
\end{itemize}

\item \textbf{ASHRAE-based Smart Home COMFORT Controller (UKY Best Senior Project 2021) -}
\begin{itemize}
\item Designed Smart HVAC at the UKY Spark Lab to meet ASHRAE "COMFORT" Standards.
\item Developed lightweight, secure communication firmware/API with embedded C, C++, and C\# for WLAN-based, smart integration for HVAC devices.
\item Documented various performance and functionality metrics for future use in Spark Lab.
\end{itemize}

\item \textbf{GUI Operating System Designed for RISC-V Architecture -} 
\begin{itemize}
\item Designed an operating system and SBI in C for a virtual RISC-V with GUI/input support.
\item Implemented supervisor, hypervisor, and user modes for privilege protection.
\item Provides "hardware threading" for a multi-core RISC-V ISA with power saving capabilities.
\end{itemize}

\item \textbf{Smart-Home Sensor Anomaly Detection using Keras Deep Learning -} 
\begin{itemize}
\item Proposed an ML model to identify sensor anomalies in Honda US Smart Home data.
\item Trained a deep learning autoencoder model in Python in collaboration with Schneider Electric.
\end{itemize}

\item \textbf{16-bit Pipelined Processor for 8-bit SIMD Posit Arithmetic in Verilog (Gr8BOnd) -} 
\begin{itemize}
\item Implemented a 16-bit pipelined processor capable of performing SIMD Within a Register (SWAR) in a Turing-complete instruction set.
\item Fully developed in Verilog and implemented on the Intel Cyclone V SoC/FPGA.
\item Developed a Monte Carlo-based testing apparatus for validating processor functionality via Icarus.
\end{itemize}

\item \textbf{Quantum Anomaly Detection for Industrial Control Systems -} 
\begin{itemize}
\item Engineered high-accuracy Quantum ML models in Python for detecting cyberattacks in critical infrastructures with Oak Ridge National Lab (ORNL).
\item Determined metrics for parametrizing the success, or advantage, of a CPS-related QSVM model on detecting real-world industrial cyberattacks.
\item Investigated the effects of modern NISQ noise in quantum machine learning.
\item Designed Python framework for easy integration of IBM Qiskit in Cyberphysical Systems.
\end{itemize}

\item \textbf{FPGA-Based Multilayer Perceptron (MLP) as a Neural Network using Verilog -}
\begin{itemize}
\item Designed a MLP neural network accelerator with parametrizable neurons and layers using dedicated DSP hardware on the Intel Cyclone V and Xilinx Artix-7.
\item Implemented wrapper/controller for easy interfacing with SoC RAM and I/O.
\item Evaluated various performance metrics of the FPGA design in terms of power, area, and timing.
\end{itemize}

\item \textbf{FPGA Implementation and Side-Channel Analysis of PRESENT \& AES Cryptography -} 
\begin{itemize}
\item Designed FPGA implementation of the PRESENT \& AES cryptographic engines with a state machine-based controller in Verilog.
\item Taught students how to perform validation and investigate power usage, timing, and utilization.
\item Performed side-channel power analysis attacks on HW implementations of PRESENT \& AES.
\end{itemize}
\end{itemize}

\section{MAJOR\\ACADEMIC\\COURSES}
\begin{itemize}
\item \bf Systems Programming,
\bf Computer Programming,
\bf Network/Software/Embedded Security,
\bf System on Chip Design,
\bf Machine Learning,
\bf Compiler Design,
\bf Operating System Design,
\bf Emerging Computing (Quantum),
\bf GPU \& Multicore Programming,
\bf Data Mining and Analytics,
\bf Adv. Embedded Systems,
\bf Algorithm Design/Analysis,
\bf Digital Logic Design
\end{itemize}

\section{PUBLICATIONS}
\begin{itemize}

\item\textbf{Tyler Cultice}, Md. Saif Hassan Onim, Annarita Giani and Himanshu Thapliyal, \say{Anomaly Detection for Real-World Cyber-Physical Security using Quantum Hybrid Support Vector Machine}, \textit{Proceedings of IEEE Computer Society Annual Symposium on VLSI 2024 (ISVLSI 2024), Knoxville, TN, USA, July 1-3, \textbf{2024 (Best Paper Award)}}

\item \textbf{Tyler Cultice}, Joseph Clark, Himanshu Thapliyal. "A Novel Hierarchical Security Solution for Controller-Area-Network-Based 3D Printing in a Post-Quantum World." Sensors 23.24 \textbf{(2023)}: 9886.

\item \textbf{Tyler Cultice}, Joseph Clark, and Himanshu Thapliyal. "Lightweight Hierarchical Root-of-Trust Framework for CAN-Based 3D Printing Security." Proceedings of the Great Lakes Symposium on VLSI 2023. \textbf{2023}.

\item \textbf{Tyler Cultice}, and Himanshu Thapliyal. "Vulnerabilities and Attacks on CAN-Based 3D Printing/Additive Manufacturing." IEEE Consumer Electronics Magazine 13.1 \textbf{(2023)}: 54-61.

\item Chin, Jun-Cheng, Himanshu Thapliyal, and \textbf{Tyler Cultice}. "CAN Bus: The Future of Additive Manufacturing (3D Printing)." IEEE Consumer Electronics Magazine \textbf{(2022)}.

\item \textbf{Tyler Cultice}, and Himanshu Thapliyal. "PUF-based post-quantum CAN-FD framework for vehicular security." Information 13.8 \textbf{(2022)}: 382.

\item \textbf{Tyler Cultice}, Dan Ionel, and Himanshu Thapliyal. "Smart home sensor anomaly detection using convolutional autoencoder neural network." 2020 IEEE International Symposium on Smart Electronic Systems (iSES)(Formerly iNiS). IEEE, \textbf{2020}.

\item \textbf{Tyler Cultice}, Carson Labrado, and Himanshu Thapliyal. "A puf based can security framework." 2020 IEEE Computer Society Annual Symposium on VLSI (ISVLSI). IEEE, \textbf{2020}.

\end{itemize}
%\section{TEACHING \\ EXPERIENCE}

%\section{DISTINCTIONS AND AWARDS}
%\begin{itemize}
%\item \textbf{2020 - Schneider Electric Fellowship, University of Kentucky:} \\
%Smart-Home Research on Machine-Learning-based CPS Anomaly Detection in collaboration with Dr. Dan Ionel from UKY.
%\item \textbf{2021: Upsilon Pi Epsilon (UPE) Award:} \\
%IEEE Computer Society Award for Academic Success in Computer Engineering.
%\item \textbf{2022-Present - NSF Graduate Research Fellowship Program (GRFP):} \\
%Prestigious fellowship awarded to outstanding graduate students with significant contributions to STEM.
%\item \textbf{2021-present:} Graduate Student Member, Institute of Electrical and Electronics Engineers (IEEE).
%\item \textbf{2023-present:} Member, Association for Computing Machinery (ACM).
%\item \textbf{2023:} Web Chair, GLSVLSI'23
%\item \textbf{2024:} Registration and Web Chair, ISVLSI'24
%\item \textbf{Reviewer:} DAC, GLSVLSI, ISVLSI

%\end{itemize}
\end{resume}
\end{document}